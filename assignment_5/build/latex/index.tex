This directory contains the framework code that will be used in the four Open\+GL exercises.

\subsection*{Building under Linux/mac\+OS }

Inside the exercise\textquotesingle{}s top-\/level directory, execute the following commands\+: \begin{DoxyVerb}mkdir build
cd build
cmake ..
make
\end{DoxyVerb}


The last command -- i.\+e. {\ttfamily make} -- compiles the application. Rerun it whenever you have added/changed code in order to recompile.

\subsection*{Building under Windows Visual Studio }


\begin{DoxyItemize}
\item Install Visual Studio Community 2013 or later
\item You will be asked, if you want to install additional packages. Make sure that you check the c++ development option.
\item Inside the exercise\textquotesingle{}s top-\/level directory create a new {\ttfamily build} folder (C\+T\+RL + S\+H\+I\+FT + N)
\item Install \href{https://cmake.org/download/}{\tt C\+Make}
\item Start the cmake-\/gui.\+exe
\item Click {\ttfamily Browse Source} and select the exercise\textquotesingle{}s top-\/level directory
\item Click {\ttfamily Browse Build} and select the created {\ttfamily build} folder
\item Click {\ttfamily Configure} and select your Visual Studio version
\item Start Configuring.
\item If no major errors occur, click {\ttfamily Generate}
\item Start Visual Studio
\item Use {\ttfamily Open Project} to load your {\ttfamily Solar\+System.\+sln}
\item On the right, there should be the solution explorer. Find the project {\ttfamily Solar\+Viewer}, right click and choose {\ttfamily Set as Start\+Up Project}
\item Press C\+T\+RL + F5 to compile and run
\end{DoxyItemize}

\subsection*{Documentation }

You may build an H\+T\+ML documentation as long as you have \href{www.doxygen.org/}{\tt Doxygen} installed. To do so, still inside the directory {\ttfamily build}, execute the following command\+: \begin{DoxyVerb}make doc
\end{DoxyVerb}


View the documentation by opening the file {\ttfamily html/index.\+html} with any web browser / H\+T\+ML viewer. If you are into La\+TeX, navigate into the directory {\ttfamily latex} and execute the command {\ttfamily make} to create a printable version of the documentation.

\subsection*{Textures and Copyright }

All earth textures are from the \href{http://earthobservatory.nasa.gov/Features/BlueMarble/}{\tt N\+A\+SA Earth Observatory} and have been modified by Prof. Hartmut Schirmacher, Beuth Hochschule für Technik Berlin. The sun texture is from \href{http://www.solarsystemscope.com/textures}{\tt http\+://www.\+solarsystemscope.\+com/textures}. All other textures are from \href{http://textures.forrest.cz/index.php?spgmGal=maps&spgmPic=14}{\tt http\+://textures.\+forrest.\+cz/index.\+php?spgm\+Gal=maps\&spgm\+Pic=14}. The ship model if from \href{https://free3d.com}{\tt https\+://free3d.\+com}.

\subsection*{Keyboard Settings }


\begin{DoxyItemize}
\item arrow keys\+: Navigation Camera
\item W,A,S,D\+: Navigation \hyperlink{classShip}{Ship}
\item g\+: toggle greyscale
\item +/-\/\+: increase/decrease time\+\_\+step
\item y/z\+: switch mono/stereo view mode
\item 1-\/6\+: set camera to planets/sun
\item 7\+: set camera to ship
\item 8/9\+: change camera\textquotesingle{}s distance to the observed object
\item space\+: pause
\item r\+: randomize planets\textquotesingle{} positions
\item escape\+: exit viewer
\end{DoxyItemize}

\subsection*{Assignment 5\+: Transformations and Viewing }

In this assignment, you will place the planets, moon, and space ship in the Solar System scene and set up the view. All of your implementation will be done in {\ttfamily \hyperlink{solar__viewer_8cpp}{src/solar\+\_\+viewer.\+cpp}}. For detailed instructions, please see the assignment 5 handout. 